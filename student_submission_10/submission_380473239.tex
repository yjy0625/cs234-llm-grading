\documentclass[11pt]{article}
\usepackage[utf8]{inputenc} 
\usepackage[T1]{fontenc}    
\usepackage{url}            
\usepackage{booktabs}       
\usepackage{amsfonts}       
\usepackage{nicefrac}
\usepackage{microtype}      
\usepackage{fullpage}
\usepackage{subcaption}
\usepackage[most]{tcolorbox}
\usepackage[numbers]{natbib}
%\usepackage[textsize=tiny]{todonotes}
\setlength{\marginparwidth}{11ex}

\newcommand{\E}{\mathbb E}
\usepackage{wrapfig}
\usepackage{caption}

\newcommand{\theHalgorithm}{\arabic{algorithm}}

\usepackage{url}

\usepackage[utf8]{inputenc}
\usepackage{amsmath}
\usepackage{graphicx}
\usepackage{upgreek}
\usepackage{amsfonts}
\usepackage{amssymb}
\usepackage{amsthm}
\usepackage[mathscr]{euscript}
\usepackage{mathtools}
\newtheorem{thm}{Theorem}
\newtheorem{defn}{Definition}
\newtheorem{cor}{Corollary}
\newtheorem{assumption}{Assumption}
\newtheorem{lem}{Lemma}
\usepackage{xcolor}
\usepackage{nicefrac}
\usepackage{xr}
%\usepackage{chngcntr}
\usepackage{apptools}
\usepackage[page, header]{appendix}
\AtAppendix{\counterwithin{lem}{section}}
\usepackage{titletoc}
\usepackage{enumitem}
\setlist[itemize]{leftmargin=1cm}
\setlist[enumerate]{leftmargin=1cm}
\newcommand*{\fnref}[1]{\textsuperscript{\ref{#1}}}



\definecolor{DarkRed}{rgb}{0.75,0,0}
\definecolor{DarkGreen}{rgb}{0,0.5,0}
\definecolor{DarkPurple}{rgb}{0.5,0,0.5}
\definecolor{Dark}{rgb}{0.5,0.5,0}
\definecolor{DarkBlue}{rgb}{0,0,0.7}
\usepackage[bookmarks, colorlinks=true, plainpages = false, citecolor = DarkBlue, urlcolor = blue, filecolor = black, linkcolor =DarkGreen]{hyperref}
\usepackage{breakurl}
\usepackage[ruled, vlined, linesnumbered]{algorithm2e}
\newcommand\mycommfont[1]{\footnotesize\ttfamily\textcolor{blue}{#1}}
\SetCommentSty{mycommfont}

\DeclareMathOperator*{\argmin}{arg\,min}
\DeclareMathOperator*{\argmax}{arg\,max}

\allowdisplaybreaks[2]
\newcommand{\prob}{\mathbb P}
\newcommand{\Var}{\mathbb V}
\newcommand{\NN}{\mathbb N}
\newcommand{\Ex}{\mathbb E}
\newcommand{\varV}{\mathscr V}
\newcommand{\indicator}[1]{\mathbb I\{ #1 \} }
\newcommand{\statespace}{\mathcal S}
\newcommand{\actionspace}{\mathcal A}
\newcommand{\saspace}{\statespace \times \actionspace}
\newcommand{\satspace}{\mathcal Z}
\newcommand{\numsa}{\left|\saspace\right|}
\newcommand{\numsat}{\left|\satspace\right|}
\newcommand{\numS}{S}
\newcommand{\numA}{A}
\newcommand{\wmin}{w_{\min}}
\newcommand{\wminc}{w'_{\min}}
\newcommand{\range}{\operatorname{rng}}
\newcommand{\polylog}{\operatorname{polylog}}
\newcommand{\dspace}{\mathcal D}
\newcommand{\numD}{|\dspace|}
\newcommand{\numSucc}[1]{|\statespace(#1)|}
\newcommand{\succS}[1]{\statespace(#1)}

\newcommand{\reals}{\mathbb R}
\newcommand{\const}{\textrm{const.}}
\newcommand{\set}[1]{\left\{#1\right\}}
\newcommand{\llnp}{\operatorname{llnp}}
\newcommand{\defeq}{:=}
\usepackage{xspace}
\newcommand{\algname}{UBEV\xspace}

\mathtoolsset{showonlyrefs=true}

\let\temp\epsilon
\let\epsilon\varepsilon
\newcommand{\cK}{\mathcal K}
\newcommand{\cI}{\mathcal I}
\newcommand{\Pro}{\mathbb P}

\title{CS 234 Winter 2026 \\ Assignment 1 \\ Due: January 16th 2026 at 6:00 pm (PST) \\
Sunid: jiamins
}
\date{}

\begin{document}
  \maketitle

\noindent\textbf{Submission Instructions (Please follow these exactly)}

\medskip

Please typeset the written part of your homework using the template that we have provided in Latex. You will need to submit the homework in three separate parts:
\begin{itemize}
    \item \textbf{Assignment 1 Written (PDF file):} Submit a PDF version of your \LaTeX{} file.
    \item \textbf{Assignment 1 Written (Tex file):} Submit the raw \texttt{.tex} file.
    \item \textbf{Assignment 1 Coding (.zip file):} Submit your coding solutions here. There is a \texttt{Makefile} provided that will help you submit the assignment. Please run \texttt{make clean} followed by \texttt{make submit} in the terminal and submit the resulting zip file on Gradescope.
\end{itemize}




\section{Effect of Effective Horizon [8 pts]} 

Consider an agent managing inventory for a store, which is represented as an MDP. The stock level $s$ refers to the number of items currently in stock (between 0 and 10, inclusive). At any time, the agent has two actions: sell (decrease stock by one, if possible) or buy (increase stock by one, if possible).  
\begin{itemize}
\item If $s 
 > 0$ and the agent sells, it receives +1 reward for the sale and the stock level transitions to $s - 1$. If $s = 0$ nothing happens.
 \item If $s < 9$ and the agent buys, it receives no reward and the stock level transitions to $s +  1$.
 \item The owner of the store likes to see a fully stocked inventory at the end of the day, so the agent is rewarded with $+100$ if the stock level ever reaches the maximum level $s = 10$.
 \item $s = 10$ is also a terminal state and the problem ends if it is reached.
\end{itemize}

The reward function, denoted as $r(s, a, s')$, can be summarized concisely as follows:
\begin{itemize}
    \item $r(s,\text{sell}, s-1) = 1$ for $s > 0$ and $r(0,\text{sell},0) = 0$
    \item $r(s, \text{buy}, s+1) = 0$ for $s < 9$ and $r(9, \text{buy}, 10) = 100$. The last condition indicates that transitioning from $s = 9$ to $s = 10$ (fully stocked) yields $+100$ reward.
\end{itemize}

\noindent The stock level is assumed to always start at $s = 3$ at the beginning of the day. We will consider how the agent's optimal policy changes as we adjust the finite horizon $H$ of the problem. Recall that the horizon $H$ refers to a limit on the number of time steps the agent can interact with the MDP before the episode terminates, regardless of whether it has reached a terminal state. We will explore properties of the optimal policy (the policy that achieves highest episode reward) as the horizon $H$ changes.\\

\noindent Consider, for example, $H = 4$. The agent can sell for three steps, transitioning from $s = 3$ to $s = 2$ to $s = 1$ to $s = 0$ receiving rewards $+1$, $+1$, and $+1$ for each sell action. At the fourth step, the inventory is empty so it can sell or buy, receiving no reward regardless. Then the problem terminates since time has expired.

\begin{enumerate}[label=(\alph*)]
    \item Starting from the initial state $s = 3$, it possible to a choose a value of $H$ that results in the optimal policy taking both buy and sell steps during its execution? Explain why or why not. [2 pts]

    \begin{tcolorbox}[breakable]
    %%%%% Start of 1(a) %%%%%
For H equal or larger than 9, the optimal policy takes both buy and sell steps. For small H e.g. H=3, it is never possible to reach s=10, therefore only sell steps will be taken for optimal reward. For H=7 or H=8, it is possible to reach s=10 without sell steps, therefore, buy steps only are optimal. For H$\geq 9$, the agent can afford sell steps for immediate reward but will still be able to reach to s=10 eventually for the overall reward. Thus the optimal policy takes both buy and sell steps. 
    %%%%% End of 1(a) %%%%%
    \end{tcolorbox}
    \newpage

    \item In the infinite-horizon discounted setting, is it possible to choose a fixed value of $\gamma \in [0, 1)$ such that the optimal policy starting from $s = 3$ never fully stocks the inventory? You do not need to propose a specific value, but simply explain your reasoning either way. [2 pts]
    \begin{tcolorbox}[breakable]
    %%%%% Start of 1(b) %%%%%
Yes it is possible to choose a small $\gamma$ such that the optimal poicy never fully stocks. If we look at a policy that the agent directly goes to s=10, the reward is $\gamma^6100$. In comparison, if we look at a policy that the agent sells three times and stays s=0 to sell, the reward is 1 + $\gamma$ + $\gamma^2$. For a small $\gamma$ such as 0.1, the second policy yields optimal total rewards of 1.11, larger than $0.1^6*100$. The full stock reward 100 is exponentially discounted and therefore the optimal policy will not fully stock the inventory. 
    %%%%% End of 1(b) %%%%%
    \end{tcolorbox}
    \newpage

    \item Consider two versions of this inventory MDP. In the first version, the MDP is an \emph{infinite-horizon} MDP with discount factor $\gamma$. In the second version, the MDP is a \emph{finite-horizon} MDP with horizon $H$, no discount factor, and episodes that terminate after exactly $H$ time steps even if a terminal state has not been reached.  

    Does there \textbf{ever} exist a choice of $\gamma$ such that the optimal policy for the infinite-horizon MDP is the same as the optimal policy for the finite-horizon MDP with horizon $H$? If so, give a concrete example of values of $\gamma$ and $H$ for which this holds. [2 pts]
    \begin{tcolorbox}[breakable]
    %%%%% Start of 1(c) %%%%%
Yes. When $\gamma=0$ and $H=1$, the infinite-horizon MDP chooses sell policy for maximizing the immediate reward, the finite-horizon MDP chooses sell policy as well within one step. 
    %%%%% End of 1(c) %%%%%
    \end{tcolorbox}
    \newpage

    \item Using the same setup as in part (c), does there \textbf{always} exist a discount factor $\gamma$ such that the optimal policy for the infinite-horizon MDP matches the optimal policy for the finite-horizon MDP for any $H$? Briefly justify your answer in 1--2 sentences.[2 pts]
    \begin{tcolorbox}[breakable]
    %%%%% Start of 1(d) %%%%%
No. For finite-horizon MDP, when H is small such that there is no way for the agent to reach s=10, the optimal policy is generally sell. For infinite-horizon MDP, we need a small discount factor to match the policy. However, when H is large, the optimal policy of finite-horizon is generally to reach s=10. If we still have the small discount factor in infinite-horizon MDP, the heavily discounted rewarding will prevent reaching s=10 as the optimal policy. Therefore, there is no single $\gamma$ that can match policies for all H. 
    %%%%% End of 1(d) %%%%%
    \end{tcolorbox}
    \newpage

    
\end{enumerate}

\pagebreak

\section{Reward Hacking [5 pts]} 
Q1  illustrates how the particular  horizon and discount factor may lead to 
    to very different policies, even with the same reward and dynamics model. This may lead to unintentional reward hacking, where the resulting policy does not match a human stakeholder's intended outcome. This problem asks you to think about an example where reward hacking may occur, introduced by Pan, Bhatia and Steinhardt\footnote{ICLR 2022 \url{https://openreview.net/pdf?id=JYtwGwIL7ye}}. Consider designing RL for autonomous cars where the goal is to have decision policies that minimize the mean commute for all drivers (those driven by humans and those driven by AI). This reward might be tricky to specify (it depends on the destination of each car, etc) but a simpler reward (called the reward "proxy") is to maximize the mean velocity of all cars. Now consider a scenario where there is a single AI car (the red car in the figure) and many cars driven by humans (the grey car). 

    In this setting, under this simpler "proxy" reward, the optimal policy for the red (AI) car is to park and not merge onto the highway.\footnote{Interestingly, it turns out that systems that use simpler function representations may reward hack less in this example than more complex representations. See Pan, Bhatia and Steinhardt's paper "The Effects of Reward Misspecification: Mapping and Mitigating Misaligned Models" for details.}
        \begin{figure}[h]
    \centering
    \includegraphics[width=3in]{commute_time.png}
    \caption{Pan, Bhatia, Steinhardt ICLR 2022; \url{https://openreview.net/pdf?id=JYtwGwIL7ye}}
    \label{fig:commute}
\end{figure}
\begin{enumerate}[label=(\alph*)]
    \item  Explain why the optimal policy for the AI car is not to merge onto the highway. [2 pts]

    \begin{tcolorbox}[breakable]
    %%%%% Start of 2(a) %%%%%
When it merges onto the highway, it may slow down the other cars. Therefore the average velocity may be decreased. To maximize the mean velocity, the AI car chooses to not interrupt and to park, let other cars remain in high speed. 
    %%%%% End of 2(a) %%%%%
    \end{tcolorbox}
    \newpage
    \item Note this behavior is not aligned with the true reward function. Share some ideas about alternate reward functions (that are not minimizing commute) that might still be easier to optimize, but would not result in the AI car never merging. Your answer should be 2-5 sentences and can include equations: there is not a single answer and reasonable solutions will be given full credit. [3 pts]

    \begin{tcolorbox}[breakable]
    %%%%% Start of 2(b) %%%%%
One possible reward function is to add a term of total distance traveled, e.g., Reward $R = v + kS$, where v is the average velocity, S is the total distance traveled by all cars, k is a factor. In this way, the distance traveled by the AI car will be rewarded and therefore the parking behavior is prevented. 
    %%%%% End of 2(b) %%%%%
    \end{tcolorbox}
    \newpage

\end{enumerate}

\newpage

\section{Bellman Residuals and performance bounds [30 pts]}

In this problem, we will study value functions and properties of the Bellman backup operator.
\\

\noindent \textbf{Definitions:} \noindent Recall that a value function is a $|S|$-dimensional vector where $|S|$ is the number of states of the MDP. When we use the term $V$ in these expressions as an ``arbitrary value function'', we mean that $V$ is an arbitrary $|S|$-dimensional vector which need not be aligned with the definition of the MDP at all. 
On the other hand, $V^\pi$ is a value function that is achieved by some policy $\pi$ in the MDP.
For example, say the MDP has 2 states and only negative immediate rewards. $V = [1,1]$ would be a valid choice for $V$ even though this value function can never be achieved by any policy $\pi$, but we can never have a $V^\pi = [1,1]$. This distinction between $V$ and $V^\pi$ is important for this question and more broadly in reinforcement learning.
\\

\noindent \textbf{Properties of Bellman Operators:} In the first part of this problem, we will explore some general and useful properties of the Bellman backup operator, which was introduced during lecture. We know that the Bellman backup operator $B$, defined below is a contraction with the fixed point as $V^*$, the optimal value function of the MDP. The symbols have their usual meanings. $\gamma$ is the discount factor and $0 \leq \gamma < 1$. In all parts, $\|v\| = \max_{s} | v(s) |$ is the infinity norm of the vector.

\begin{equation}
    (BV)(s) = \max_a \left( r(s, a) + \gamma\sum_{s' \in S}p(s'|s,a)V(s') \right)
\end{equation}

\noindent We also saw the contraction operator $B^\pi$ with the fixed point $V^\pi$, which is the Bellman backup operator for a particular policy given below:

\begin{equation}
    (B^\pi V)(s) = r(s,\pi(s)) + \gamma\sum_{s' \in S}p(s'|s,\pi(s))V(s')
\end{equation}


\noindent In this case, we'll assume $\pi$ is deterministic, but it doesn't have to be in general. In class, we showed that $|| BV - BV' || \leq \gamma ||V - V'||$ for two arbitrary value functions $V$ and $V'$. 

\begin{enumerate}[label=(\alph*)]
    \item Show that the analogous inequality, $|| B^\pi V - B^\pi V' || \leq \gamma ||V - V'||$, also holds. [3 pts].

    \begin{tcolorbox}[breakable]
    %%%%% Start of 3(a) %%%%%
$|| B^\pi V - B^\pi V' || = ||r(s,\pi(s)) + \gamma\sum_{s' \in S}p(s'|s,\pi(s))V(s') - (r(s,\pi(s)) + \gamma\sum_{s' \in S}p(s'|s,\pi(s))V'(s'))|| = ||\gamma\sum_{s' \in S}p(s'|s,\pi(s))(V(s') - V'(s'))|| \leq ||\gamma\sum_{s' \in S}p(s'|s,\pi(s))||V(s') - V'(s')|||| \leq ||\gamma||V - V'||_\infty\sum_{s' \in S}p(s'|s,\pi(s))||= \gamma ||V - V'||_\infty$
    %%%%% End of 3(a) %%%%%
    \end{tcolorbox}
    \newpage
    \item Prove that the fixed point for $B^\pi$ is unique. Recall that the fixed point is defined as $V$ satisfying $V = B^\pi V$. You may assume that a fixed point exists. \textit{Hint:} Consider proof by contradiction. [3 pts].

    \begin{tcolorbox}[breakable]
    %%%%% Start of 3(b) %%%%%
    Let's say there are two fixed points $V_1$ and $V_2$, then $V_1 = B^\pi V_1, V_2 = B^\pi V_2$. According to (a), $|| B^\pi V_1 - B^\pi V_2 || = ||V_1 - V_2||\leq \gamma ||V_1 - V_2||$. Divide the term $||V_1 - V_2||$ from both sides ($V_1$, $V_2$ are different, this term is positive), we have $1 \leq \gamma$. This contradicts with the definition of $0 \leq \gamma < 1$. 
    %%%%% End of 3(b) %%%%%
    \end{tcolorbox}
    \newpage
    \item Suppose that $V$ and $V'$ are vectors satisfying $V(s) \leq V'(s)$ for all $s$. Show that $B^\pi V(s) \leq B^\pi V'(s)$ for all $s$. Note that all of these inequalities are element-wise. [3 pts].
    
    \begin{tcolorbox}[breakable]
    %%%%% Start of 3(c) %%%%%
    For $s' \in S$, $V(s') \leq V'(s') \rightarrow \gamma V(s') \leq \gamma V'(s') \rightarrow $
    
    $\gamma\sum_{s' \in S}p(s'|s,\pi(s))V(s') \leq \gamma\sum_{s' \in S}p(s'|s,\pi(s))V'(s') \rightarrow$

    $r(s,\pi(s))+\gamma\sum_{s' \in S}p(s'|s,\pi(s))V(s') \leq r(s,\pi(s))+\gamma\sum_{s' \in S}p(s'|s,\pi(s))V'(s')\rightarrow$

    $B^\pi V(s) \leq B^\pi V'(s)$
    %%%%% End of 3(c) %%%%%
    \end{tcolorbox}
    \newpage
\end{enumerate}


\noindent \textbf{Bellman Residuals:} Having gained some intuition for value functions and the Bellman operators, we now turn to understanding how policies can be extracted and what their performance might look like. We can extract a greedy policy $\pi$ from an arbitrary value function $V$ using the equation below. 
\begin{equation}
    \pi(s) = \argmax_{a} [{r(s,a) + \gamma\sum_{s' \in S}p(s'|s,a)V(s')}]
\end{equation}

It is often helpful to know what the performance will be if we extract a greedy policy from an arbitrary value function. To see this, we introduce the notion of a Bellman residual.

Define the Bellman residual to be $(BV - V)$ and the Bellman error magnitude to be $||BV - V||$.

\begin{enumerate}
    \item[(d)] For what value function $V$ does the Bellman error magnitude $\|BV - V \|$ equal 0? Why? [2 pts]

    \begin{tcolorbox}[breakable]
    %%%%% Start of 3(d) %%%%%
For optimal value $V^*$. Reason: $||BV-V||=0, BV=V$ meaning $V = V^*$ which is the unqiue fixed point. 
    %%%%% End of 3(d) %%%%%
    \end{tcolorbox}
    \newpage
    \item[(e)] Prove the following statements for an arbitrary value function $V$ and any policy $\pi$.  [5 pts]\\
    \textit{Hint:} Try leveraging the triangle inequality by inserting a zero term.
    \begin{equation}
        ||V - V^\pi|| \leq \frac{||V - B^\pi V||}{1-\gamma}
    \end{equation}
    \begin{equation}
        ||V - V^*|| \leq \frac{||V - BV||}{1-\gamma}
    \end{equation}

    \begin{tcolorbox}[breakable]
    %%%%% Start of 3(e) %%%%%
First, $ ||V - V^\pi|| = ||V-B^\pi V^\pi|| = ||V-B^\pi V+B^\pi V -B^\pi V^\pi|| \leq ||V-B^\pi V||+||B^\pi V -B^\pi V^\pi||\leq ||V-B^\pi V|| + \gamma||V-V^\pi|| \rightarrow$

$||V - V^\pi|| \leq \frac{||V - B^\pi V||}{1-\gamma}$


Second, $ ||V - V^*|| = ||V-BV^*|| = ||V-BV+BV -BV^*|| \leq ||V-BV||+||BV -BV^*||\leq ||V-BV|| + \gamma||V-V^*|| \rightarrow$

$||V - V^*|| \leq \frac{||V - BV||}{1-\gamma}$
    %%%%% End of 3(e) %%%%%
    \end{tcolorbox}
    \newpage
\end{enumerate}

\noindent The result you proved in part (e) will be useful in proving a bound on the policy performance in the next few parts. Given the Bellman residual, we will now try to derive a bound on the policy performance, $V^\pi$.

\begin{enumerate}
    \item[(f)] Let $V$ be an arbitrary value function and $\pi$ be the greedy policy extracted from $V$. Let $\epsilon = ||BV-V||$ be the Bellman error magnitude for $V$. Prove the following for any state $s$. [5 pts]\\
    \textit{Hint:} Try to use the results from part (e).
    \begin{equation}
        V^\pi(s) \geq V^*(s) - \frac{2\epsilon}{1-\gamma}
    \end{equation}

    \begin{tcolorbox}[breakable]
    %%%%% Start of 3(f) %%%%%
$\pi$ is greedy, therefore, $B^\pi V = BV$.

$||V^* - V^\pi|| = ||V^* - V + V - V^\pi|| \leq ||V^* - V|| + ||V - V^\pi|| \leq \frac{||V - BV||}{1-\gamma} + \frac{||V - B^\pi V||}{1-\gamma} = \frac{2\epsilon}{1-\gamma} \rightarrow$

$V^* (s) - V^\pi (s) \leq ||V^* - V^\pi|| \leq \frac{2\epsilon}{1-\gamma} \rightarrow$

$V^\pi(s) \geq V^*(s) - \frac{2\epsilon}{1-\gamma}$
    %%%%% End of 3(f) %%%%%
    \end{tcolorbox}
    \newpage
    \item[(g)] Give an example real-world application or domain where having a lower bound on $V^\pi(s)$ would be useful. [2 pt]
    
    \begin{tcolorbox}[breakable]
    %%%%% Start of 3(g) %%%%%
  In autonomous driving, this lower bound on $V^\pi(s)$ ensures the worst-case performance of a greedy policy by regulating the distance from the optimal policy. Therefore, safety requirement such as lower the overspeed risk can be ensured. 
    %%%%% End of 3(g) %%%%%
    \end{tcolorbox}\newpage

    \item[(h)] Suppose we have another value function $V'$ and extract its greedy policy $\pi'$.  $\|B V' - V' \| = \epsilon = \|B V - V\|$. Does the above lower bound imply that $V^\pi(s) = V^{\pi'}(s)$ at any $s$? [2 pts]
    
    \begin{tcolorbox}[breakable]
    %%%%% Start of 3(h) %%%%%
No. The same lower bound doesn't indicate $V^\pi(s) = V^{\pi'}(s)$ because actual performance can still vary differently between the optimal value and the lower bound.  
    %%%%% End of 3(h) %%%%%
    \end{tcolorbox}

\end{enumerate}


\newpage
\noindent {A little bit more notation:} define $V \leq V'$ if $\forall s$, $V(s) \leq V'(s)$. 
\\

\noindent What if our algorithm returns a $V$ that satisfies $V^* \leq V$? I.e., it returns a value function that is better than the optimal value function of the MDP. Once again, remember that $V$ can be any vector, not necessarily achievable in the MDP but we would still like to bound the performance of $V^\pi$ where $\pi$ is extracted from said $V$. We will show that if this condition is met, then we can achieve an even tighter bound on policy performance.



\begin{enumerate}
    \item[(i)] Using the same notation and setup as part (e), if $V^* \leq V$, show the following holds for any state $s$. [5 pts]\\
    \textit{Hint:} Recall that $\forall \pi$, $V^\pi \leq V^*$. (why?)
    \begin{equation}
        V^\pi(s) \geq V^*(s) - \frac{\epsilon}{1-\gamma}
    \end{equation}
     where $\epsilon = \|B V - V\|$  (as above) and the policy $\pi$ is the greedy policy induced by $V$. 
    
    \begin{tcolorbox}[breakable]
    %%%%% Start of 3(i) %%%%%
$V - V^\pi = V - B^\pi V^\pi = V - B^\pi V + B^\pi V - B^\pi V^\pi \rightarrow ||V - V^\pi|| \leq ||V - B^\pi V|| + ||B^\pi V - B^\pi V^\pi|| = ||V - BV|| + ||B^\pi V - B^\pi V^\pi|| \leq \epsilon + \gamma||V - V^\pi||$.

Therefore, we have $||V - V^\pi|| \leq \frac{\epsilon}{1-\gamma}, V - V^\pi \leq \frac{\epsilon}{1-\gamma}$

$V^* - V^\pi = V^* - V + V - V^\pi, V^* \leq V \rightarrow V^* - V^\pi \leq V - V^\pi \leq \frac{\epsilon}{1-\gamma}$.
This shows for an arbitrary s, $V^*(s) - V^\pi(s) \leq \frac{\epsilon}{1-\gamma} \rightarrow V^\pi(s) \geq V^*(s) - \frac{\epsilon}{1-\gamma}$.
    %%%%% End of 3(i) %%%%%
    \end{tcolorbox}
\end{enumerate}

\noindent \textbf{Intuition:} A useful way to interpret the results from parts (h) (and (i)) is based on the observation that a constant immediate reward of $r$ at every time-step leads to an overall discounted reward of $r + \gamma r + \gamma^2 r + \ldots = \frac{r}{1-\gamma}$. Thus, the above results say that a state value function $V$ with Bellman error magnitude $\epsilon$ yields a greedy policy whose reward per step (on average), differs from optimal by at most $2\epsilon$. So, if we develop an algorithm that reduces the Bellman residual, we're also able to bound the performance of the policy extracted from the value function outputted by that algorithm, which is very useful!
\\
\newpage
\noindent \textbf{Challenges:} Try to prove the following if you're interested. \textbf{These parts will not be graded.}

\begin{enumerate}
    \item[(j)] It's not easy to show that the condition $V^* \leq V$ holds because we often don't know $V^*$ of the MDP. Show that if $BV \leq V$ then $V^* \leq V$. Note that this sufficient condition is much easier to check and does not require knowledge of $V^*$. \\
    \textit{Hint}: Try to apply induction. What is $\lim\limits_{n \rightarrow \infty} B^n V$? 
    
    \begin{tcolorbox}[breakable]
    %%%%% Start of 3(j) %%%%%

    %%%%% End of 3(j) %%%%%
    \end{tcolorbox}
\newpage
\item[(k)] It is possible to make the bounds from parts (f) and (i) tighter. 
Let $V$ be an arbitrary value function and $\pi$ be the greedy policy extracted from $V$. Let $\epsilon = ||BV-V||$ be the Bellman error magnitude for $V$. Prove the following for any state $s$:
\begin{equation}
        V^\pi(s) \geq V^*(s) - \frac{2\gamma\epsilon}{1-\gamma}
\end{equation}
Further, if $V^* \leq V$, prove for any state $s$
\begin{equation}
        V^\pi(s) \geq V^*(s) - \frac{\gamma\epsilon}{1-\gamma}
\end{equation} 
    
    \begin{tcolorbox}[breakable]
    %%%%% Start of 3(k) %%%%%

    %%%%% End of 3(k) %%%%%
    \end{tcolorbox}
\end{enumerate}
\newpage

\noindent 


\newpage

\section{RiverSwim MDP [25 pts]}
Now you will implement value iteration and policy iteration for the RiverSwim environment (see picture below\footnote{Figure copied from \href{https://proceedings.neurips.cc/paper/2013/hash/6a5889bb0190d0211a991f47bb19a777-Abstract.html}{(Osband \& Van Roy, 2013)}.}) of \href{https://www.sciencedirect.com/science/article/pii/S0022000008000767}{(Strehl \& Littman, 2008)}.
\begin{figure}[h]
    \centering
    \includegraphics[width=\linewidth]{RiverSwim.png}
    \caption{The RiverSwim MDP where dashed and solid arrows represent transitions for the \textsc{LEFT} and \textsc{RIGHT} actions, respectively. The assignment uses a modified, customizable version of what is shown above where there are three different strengths (\textsc{WEAK}, \textsc{MEDIUM}, or \textsc{STRONG}) of the current (transition probabilities for being pushed back or successfully swimming \textsc{RIGHT}).}
    \label{fig:riverswim}
\end{figure}

\noindent \textbf{Setup:} This assignment needs to be completed with Python3 and \texttt{numpy}. 
\\

\noindent \textbf{Submission:} There is a \texttt{Makefile} provided that will help you submit the assignment. Please run \texttt{make clean} followed by \texttt{make submit} in the terminal and submit the resulting zip file on Gradescope.

\begin{enumerate}[label=(\alph*)]
\item \textbf{(coding)} Read through \texttt{vi\_and\_pi.py} and implement \texttt{bellman\_backup}. Return the value associated with a single Bellman backup performed for an input state-action pair. [4 pts]\\


\item \textbf{(coding)} Implement \texttt{policy\_evaluation}, \texttt{policy\_improvement} and \texttt{policy\_iteration} in  \texttt{vi\_and\_pi.py}. Return the optimal value function and the optimal policy. [8pts]\\


\item \textbf{(coding)} Implement \texttt{value\_iteration} in \texttt{vi\_and\_pi.py}. Return the optimal value function and the optimal policy. [8 pts]\\


\item \textbf{(written)} Run both methods on RiverSwim with a \textsc{weak} current strength and find the largest discount factor (\textbf{only} up to two decimal places) such that an optimal agent starting in the initial far-left state (state $s_1$ in Figure \ref{fig:riverswim}) \textbf{does not} swim up the river (that is, does not go \textsc{RIGHT}). Using the value you find, interpret why this behavior makes sense. Now repeat this for RiverSwim with \textsc{medium} and \textsc{strong} currents, respectively. Describe and explain the changes in optimal values and discount factors you obtain both quantitatively and qualitatively. [5 pts]\\ \\
\textit{Sanity Check:} For RiverSwim with a discount factor $\gamma = 0.99$ and a \textsc{weak} current, the values for the left-most and right-most states ($s_1$ and $s_6$ in Figure \ref{fig:riverswim} above) are \texttt{30.328} and \texttt{36.859} when computed with a tolerance of $0.001$. The value functions from VI and PI should be within error tolerance $0.001$ of these values. You can use this to verify your implementation. For grading purposes, we shall test your implementation against other hidden test cases as well.
 
\begin{tcolorbox}[breakable]
%%%%% Start of 4(d) %%%%%
With weak current, the largest discount factor such that agent does not go RIGHT on $s_1$ is 0.67. The values are $[0.015, 0.033, 0.092, 0.257, 0.717, 1.993]$. 0.67 puts a heavy discount on future reward. Agent favors immediate reward even with a weak current. 

Medium: Discount factor found is 0.77, values $[0.022, 0.035, 0.095, 0.275, 0.798, 2.314]$. Values increase slightly compared with weak current due to increased future reward. Swimming right has less successful rate, therefore, the future reward needs to be discounted less in order for the agent to take the risk. 

Strong: Discount factor found is 0.93. values $[0.068, 0.078, 0.146, 0.377, 1.079, 3.169]$. Value increase compared with weak and medium currents due to increased future reward. The strong current introduces the lowest successful rate for moving right. The highest discount factor is required for the agent to be willing to try longer and takes the future reward into consideration. 
%%%%% End of 4(d) %%%%%
\end{tcolorbox}

\end{enumerate}

\end{document} 