\documentclass[11pt]{article}
\usepackage[utf8]{inputenc} 
\usepackage[T1]{fontenc}    
\usepackage{url}            
\usepackage{booktabs}       
\usepackage{amsfonts}       
\usepackage{nicefrac}
\usepackage{microtype}      
\usepackage{fullpage}
\usepackage{subcaption}
\usepackage[most]{tcolorbox}
\usepackage[numbers]{natbib}
%\usepackage[textsize=tiny]{todonotes}
\setlength{\marginparwidth}{11ex}

\newcommand{\E}{\mathbb E}
\usepackage{wrapfig}
\usepackage{caption}

\newcommand{\theHalgorithm}{\arabic{algorithm}}

\usepackage{url}

\usepackage[utf8]{inputenc}
\usepackage{amsmath}
\usepackage{graphicx}
\usepackage{upgreek}
\usepackage{amsfonts}
\usepackage{amssymb}
\usepackage{amsthm}
\usepackage[mathscr]{euscript}
\usepackage{mathtools}
\newtheorem{thm}{Theorem}
\newtheorem{defn}{Definition}
\newtheorem{cor}{Corollary}
\newtheorem{assumption}{Assumption}
\newtheorem{lem}{Lemma}
\usepackage{xcolor}
\usepackage{nicefrac}
\usepackage{xr}
%\usepackage{chngcntr}
\usepackage{apptools}
\usepackage[page, header]{appendix}
\AtAppendix{\counterwithin{lem}{section}}
\usepackage{titletoc}
\usepackage{enumitem}
\setlist[itemize]{leftmargin=1cm}
\setlist[enumerate]{leftmargin=1cm}
\newcommand*{\fnref}[1]{\textsuperscript{\ref{#1}}}



\definecolor{DarkRed}{rgb}{0.75,0,0}
\definecolor{DarkGreen}{rgb}{0,0.5,0}
\definecolor{DarkPurple}{rgb}{0.5,0,0.5}
\definecolor{Dark}{rgb}{0.5,0.5,0}
\definecolor{DarkBlue}{rgb}{0,0,0.7}
\usepackage[bookmarks, colorlinks=true, plainpages = false, citecolor = DarkBlue, urlcolor = blue, filecolor = black, linkcolor =DarkGreen]{hyperref}
\usepackage{breakurl}
\usepackage[ruled, vlined, linesnumbered]{algorithm2e}
\newcommand\mycommfont[1]{\footnotesize\ttfamily\textcolor{blue}{#1}}
\SetCommentSty{mycommfont}

\DeclareMathOperator*{\argmin}{arg\,min}
\DeclareMathOperator*{\argmax}{arg\,max}

\allowdisplaybreaks[2]
\newcommand{\prob}{\mathbb P}
\newcommand{\Var}{\mathbb V}
\newcommand{\NN}{\mathbb N}
\newcommand{\Ex}{\mathbb E}
\newcommand{\varV}{\mathscr V}
\newcommand{\indicator}[1]{\mathbb I\{ #1 \} }
\newcommand{\statespace}{\mathcal S}
\newcommand{\actionspace}{\mathcal A}
\newcommand{\saspace}{\statespace \times \actionspace}
\newcommand{\satspace}{\mathcal Z}
\newcommand{\numsa}{\left|\saspace\right|}
\newcommand{\numsat}{\left|\satspace\right|}
\newcommand{\numS}{S}
\newcommand{\numA}{A}
\newcommand{\wmin}{w_{\min}}
\newcommand{\wminc}{w'_{\min}}
\newcommand{\range}{\operatorname{rng}}
\newcommand{\polylog}{\operatorname{polylog}}
\newcommand{\dspace}{\mathcal D}
\newcommand{\numD}{|\dspace|}
\newcommand{\numSucc}[1]{|\statespace(#1)|}
\newcommand{\succS}[1]{\statespace(#1)}

\newcommand{\reals}{\mathbb R}
\newcommand{\const}{\textrm{const.}}
\newcommand{\set}[1]{\left\{#1\right\}}
\newcommand{\llnp}{\operatorname{llnp}}
\newcommand{\defeq}{:=}
\usepackage{xspace}
\newcommand{\algname}{UBEV\xspace}

\mathtoolsset{showonlyrefs=true}

\let\temp\epsilon
\let\epsilon\varepsilon
\newcommand{\cK}{\mathcal K}
\newcommand{\cI}{\mathcal I}
\newcommand{\Pro}{\mathbb P}

\title{CS 234 Winter 2025 \\ Assignment 1 \\ Due: January 17th 2025 at 6:00 pm (PST) \\
Sunid: jzcheng
}
\date{}

\begin{document}
  \maketitle
%\noindent For submission instructions, please refer to the \href{http://web.stanford.edu/class/cs234/assignments.html}{website}.
%For all problems, if you use an existing result from either the literature or a textbook to solve the exercise, you need to cite the source.


\section{Effect of Effective Horizon [8 pts]} 

Consider an agent managing inventory for a store, which is represented as an MDP. The stock level $s$ refers to the number of items currently in stock (between 0 and 10, inclusive). At any time, the agent has two actions: sell (decrease stock by one, if possible) or buy (increase stock by one, if possible).  
\begin{itemize}
\item If $s 
 > 0$ and the agent sells, it receives +1 reward for the sale and the stock level transitions to $s - 1$. If $s = 0$ nothing happens.
 \item If $s < 9$ and the agent buys, it receives no reward and the stock level transitions to $s +  1$.
 \item The owner of the store likes to see a fully stocked inventory at the end of the day, so the agent is rewarded with $+100$ if the stock level ever reaches the maximum level $s = 10$.
 \item $s = 10$ is also a terminal state and the problem ends if it is reached.
\end{itemize}

The reward function, denoted as $r(s, a, s')$, can be summarized concisely as follows:
\begin{itemize}
    \item $r(s,\text{sell}, s-1) = 1$ for $s > 0$ and $r(0,\text{sell},0) = 0$
    \item $r(s, \text{buy}, s+1) = 0$ for $s < 9$ and $r(9, \text{buy}, 10) = 100$. The last condition indicates that transitioning from $s = 9$ to $s = 10$ (fully stocked) yields $+100$ reward.
\end{itemize}

\noindent The stock level is assumed to always start at $s = 3$ at the beginning of the day. We will consider how the agent's optimal policy changes as we adjust the finite horizon $H$ of the problem. Recall that the horizon $H$ refers to a limit on the number of time steps the agent can interact with the MDP before the episode terminates, regardless of whether it has reached a terminal state. We will explore properties of the optimal policy (the policy that achieves highest episode reward) as the horizon $H$ changes.\\

\noindent Consider, for example, $H = 4$. The agent can sell for three steps, transitioning from $s = 3$ to $s = 2$ to $s = 1$ to $s = 0$ receiving rewards $+1$, $+1$, and $+1$ for each sell action. At the fourth step, the inventory is empty so it can sell or buy, receiving no reward regardless. Then the problem terminates since time has expired.

\begin{enumerate}[label=(\alph*)]
    \item Starting from the initial state $s = 3$, it possible to a choose a value of $H$ that results in the optimal policy taking both buy and sell steps during its execution? Explain why or why not. [2 pts]

    \begin{tcolorbox}[breakable]
    %%%%% Start of 1(a) %%%%%
    Yes, it is possible to take both buy and sell steps. For instance, if we set $H=9$, then the optimal policy will result in: $s_0 = 3$, $s_1=2$, $s_2=3$, $s_3=4$, $s_4=5$, $s_5=6$, $s_6=7$, $s_7=8$, $s_8=9$, $s_9=10$, resulting in a reward of 101. Essentially, if the time horizon is long enough, the agent will keep selling / buying, and then it will restock up to 10.
    %%%%% End of 1(a) %%%%%
    \end{tcolorbox}
    \newpage

    \item In the infinite-horizon discounted setting, is it possible to choose a fixed value of $\gamma \in [0, 1)$ such that the optimal policy starting from $s = 3$ never fully stocks the inventory? You do not need to propose a specific value, but simply explain your reasoning either way. [2 pts]

    \begin{tcolorbox}[breakable]
    %%%%% Start of 1(b) %%%%%
    Yes, it is possible such that the optimal policy never fully stocks the inventory in the infinite-horizon discounted setting if we set $\gamma=0$. In this situation, we only care about the immediate reward, so the agent will sell whenever it can, preventing it from fully stocking the inventory.
    %%%%% End of 1(b) %%%%%
    \end{tcolorbox}
    \newpage

    \item Does there \textbf{ever} exist a $\gamma$ such that the optimal policy for a MDP with a gamma is the same as a MDP with a finite horizon $H$? Please give an example of a particular $\gamma$ if there exists one. [2 pts]

    \begin{tcolorbox}[breakable]
    %%%%% Start of 1(c) %%%%%
    For $\gamma=0$, we only care about the immediate reward in both the infinite-horizon MDP and finite-horizon MDP, so the optimal policies would be the same.
    %%%%% End of 1(c) %%%%%
    \end{tcolorbox}
    \newpage

    \item Does there \textbf{always} exist a $\gamma$ such that the optimal policy for the MDP with $\gamma$ is the same as an MDP with finite horizon $H$? Please provide a discussion (1-2 sentences) describing your reasoning. [2 pts]

    \begin{tcolorbox}[breakable]
    %%%%% Start of 1(d) %%%%%
    \textbf{No.} A discount factor $\gamma$ does not always exist such that the infinite-horizon policy matches the finite-horizon policy.
    
    The optimal policy for a finite-horizon MDP is generally \textit{non-stationary} (time-dependent), defined as $\pi(s, t)$, whereas the optimal policy for a discounted infinite-horizon MDP is always \textit{stationary} (static), defined as $\pi(s)$. A stationary policy must prescribe a single, consistent action for a given state $s$, and therefore cannot replicate a strategy that changes actions for the same state based on the remaining time.
    
    \textbf{Counter-Example:} Consider the inventory MDP starting at $s=9$ with horizon $H=3$.
    \begin{itemize}
        \item \textbf{Finite Horizon ($H=3$):} The agent has enough time to ``milk'' a sale before finishing. 
        \begin{enumerate}
            \item Step 1 ($s=9$): Action \textbf{Sell} (+1 reward, transition to $s=8$).
            \item Step 2 ($s=8$): Action \textbf{Buy} (0 reward, transition to $s=9$).
            \item Step 3 ($s=9$): Action \textbf{Buy} (+100 reward, transition to $s=10$).
        \end{enumerate}
        Total Reward: 101. Note that the agent takes action \textbf{Sell} at $s=9$ (Step 1) but later takes action \textbf{Buy} at $s=9$ (Step 3).
        
        \item \textbf{Infinite Horizon:} A stationary policy $\pi(s)$ must pick exactly one action for $s=9$. If $\pi(9) = \text{Buy}$, it misses the initial +1 sale (Total 100). If $\pi(9) = \text{Sell}$, it enters a loop $9 \to 8 \to \dots$ or never reaches the terminal state efficiently to get the +100.
    \end{itemize}
    It cannot replicate the ``Sell then Buy'' behavior required by the optimal finite policy.
    %%%%% End of 1(d) %%%%%
    \end{tcolorbox}
    \newpage

    
\end{enumerate}

\pagebreak

\section{Reward Hacking [5 pts]} 
Q1  illustrates how the particular  horizon and discount factor may lead to 
    to very different policies, even with the same reward and dynamics model. This may lead to unintentional reward hacking, where the resulting policy does not match a human stakeholder's intended outcome. This problem asks you to think about an example where reward hacking may occur, introduced by Pan, Bhatia and Steinhardt\footnote{ICLR 2022 \url{https://openreview.net/pdf?id=JYtwGwIL7ye}}. Consider designing RL for autonomous cars where the goal is to have decision policies that minimize the mean commute for all drivers (those driven by humans and those driven by AI). This reward might be tricky to specify (it depends on the destination of each car, etc) but a simpler reward (called the reward "proxy") is to maximize the mean velocity of all cars. Now consider a scenario where there is a single AI car (the red car in the figure) and many cars driven by humans (the grey car). 

    In this setting, under this simpler "proxy" reward, the optimal policy for the red (AI) car is to park and not merge onto the highway.\footnote{Interestingly, it turns out that systems that use simpler function representations may reward hack less in this example than more complex representations. See Pan, Bhatia and Steinhardt's paper "The Effects of Reward Misspecification: Mapping and Mitigating Misaligned Models" for details.}
        \begin{figure}[h]
    \centering
    \includegraphics[width=3in]{commute_time.png}
    \caption{Pan, Bhatia, Steinhardt ICLR 2022; \url{https://openreview.net/pdf?id=JYtwGwIL7ye}}
    \label{fig:commute}
\end{figure}
\begin{enumerate}[label=(\alph*)]
    \item  Explain why the optimal policy for the AI car is not to merge onto the highway. [2 pts]

    \begin{tcolorbox}[breakable]
    %%%%% Start of 2(a) %%%%%
    Our reward proxy is to maximize the mean velocity of all cars; the mean is mostly impacted by the grey cars because there are more grey cars. Therefore, the agent would naturally want the grey cars to have higher velocity, resulting in the AI car to not merge. If the AI car did merge, the mean velocity would decrease.
    %%%%% End of 2(a) %%%%%
    \end{tcolorbox}
    \newpage
    \item Note this behavior is not aligned with the true reward function. Share some ideas about alternate reward functions (that are not minimizing commute) that might still be easier to optimize, but would not result in the AI car never merging. Your answer should be 2-5 sentences and can include equations: there is not a single answer and reasonable solutions will be given full credit. [3 pts]

    \begin{tcolorbox}[breakable]
    %%%%% Start of 2(b) %%%%%
    One proxy reward we can consider is to add a penalty for any car that is stationary, i.e. $R = \alpha \times (\text{mean velocity of all cars}) - \beta \times (\text{count of stationary cars})$. The stationary cars can also be weighted by how long they have been stationary, i.e. cars stationary for a longer time are weighted more. This reward still aims for the overarching goal of minimizing mean commute time and ensures that the AI car would eventually merge because of the penalty (we can scale the relative rewards and penalties using $\alpha$ and $\beta$).
    %%%%% End of 2(b) %%%%%
    \end{tcolorbox}
    \newpage

\end{enumerate}

\newpage

\section{Bellman Residuals and performance bounds [30 pts]}

In this problem, we will study value functions and properties of the Bellman backup operator.
\\

\noindent \textbf{Definitions:} \noindent Recall that a value function is a $|S|$-dimensional vector where $|S|$ is the number of states of the MDP. When we use the term $V$ in these expressions as an ``arbitrary value function'', we mean that $V$ is an arbitrary $|S|$-dimensional vector which need not be aligned with the definition of the MDP at all. 
On the other hand, $V^\pi$ is a value function that is achieved by some policy $\pi$ in the MDP.
For example, say the MDP has 2 states and only negative immediate rewards. $V = [1,1]$ would be a valid choice for $V$ even though this value function can never be achieved by any policy $\pi$, but we can never have a $V^\pi = [1,1]$. This distinction between $V$ and $V^\pi$ is important for this question and more broadly in reinforcement learning.
\\

\noindent \textbf{Properties of Bellman Operators:} In the first part of this problem, we will explore some general and useful properties of the Bellman backup operator, which was introduced during lecture. We know that the Bellman backup operator $B$, defined below is a contraction with the fixed point as $V^*$, the optimal value function of the MDP. The symbols have their usual meanings. $\gamma$ is the discount factor and $0 \leq \gamma < 1$. In all parts, $\|v\| = \max_{s} | v(s) |$ is the infinity norm of the vector.

\begin{equation}
    (BV)(s) = \max_a \left( r(s, a) + \gamma\sum_{s' \in S}p(s'|s,a)V(s') \right)
\end{equation}

\noindent We also saw the contraction operator $B^\pi$ with the fixed point $V^\pi$, which is the Bellman backup operator for a particular policy given below:

\begin{equation}
    (B^\pi V)(s) = r(s,\pi(s)) + \gamma\sum_{s' \in S}p(s'|s,\pi(s))V(s')
\end{equation}


\noindent In this case, we'll assume $\pi$ is deterministic, but it doesn't have to be in general. In class, we showed that $|| BV - BV' || \leq \gamma ||V - V'||$ for two arbitrary value functions $V$ and $V'$. 

\begin{enumerate}[label=(\alph*)]
    \item Show that the analogous inequality, $|| B^\pi V - B^\pi V' || \leq \gamma ||V - V'||$, also holds. [3 pts].

    \begin{tcolorbox}[breakable]
    %%%%% Start of 3(a) %%%%%
    Starting from the LHS,
    \begin{align*}
        || B^\pi V - B^\pi V'|| &= \max_s | r(s, \pi(s)) + \gamma \sum_{s' \in S} p(s' | s, \pi(s)) V(s') - r(s, \pi(s)) - \gamma \sum_{s' \in S} p(s' | s, \pi(s)) V'(s')| \\
        &= \max_s |\gamma \sum_{s' \in S} p(s' | s, \pi(s))(V(s') - V'(s')) | \\
        &\leq \max_s |\gamma \sum_{s' \in S} p(s' | s, \pi(s)) ||V-V'|| | \\
        &= \gamma ||V-V'||
    \end{align*}
    %%%%% End of 3(a) %%%%%
    \end{tcolorbox}
    \newpage
    \item Prove that the fixed point for $B^\pi$ is unique. Recall that the fixed point is defined as $V$ satisfying $V = B^\pi V$. You may assume that a fixed point exists. \textit{Hint:} Consider proof by contradiction. [3 pts].

    \begin{tcolorbox}[breakable]
    %%%%% Start of 3(b) %%%%%
    Assume we have two fixed points $V$ and $V'$ such that $V = B^\pi V$, $V' = B^\pi V'$, and $V \neq V'$. Then, subtracting the two inequalities and applying the norm, we have 
    \begin{align*}
        || V-V' || = || B^\pi V - B^\pi V' || \leq \gamma ||V-V'|| 
    \end{align*}
    where the last inequality is from part (a). However, this is a contradiction / impossible because $\gamma \in [0, 1)$! A similar logic can be applied for more than two fixed points. Therefore, the fixed point for $B^\pi$ must be unique.
    %%%%% End of 3(b) %%%%%
    \end{tcolorbox}
    \newpage
    \item Suppose that $V$ and $V'$ are vectors satisfying $V(s) \leq V'(s)$ for all $s$. Show that $B^\pi V(s) \leq B^\pi V'(s)$ for all $s$. Note that all of these inequalities are element-wise. [3 pts].
    
    \begin{tcolorbox}[breakable]
    %%%%% Start of 3(c) %%%%%
    For all $s$, 
    \begin{align*}
        B^\pi V(s) &= r(s, \pi(s)) + \gamma \sum_{s' \in S} p(s' | s, \pi(s)) V(s') \\
        &\leq r(s, \pi(s)) + \gamma \sum_{s' \in S} p(s' | s, \pi(s)) V'(s') \\
        &= B^\pi V'(s)
    \end{align*}
    %%%%% End of 3(c) %%%%%
    \end{tcolorbox}
    \newpage
\end{enumerate}


\noindent \textbf{Bellman Residuals:} Having gained some intuition for value functions and the Bellman operators, we now turn to understanding how policies can be extracted and what their performance might look like. We can extract a greedy policy $\pi$ from an arbitrary value function $V$ using the equation below. 
\begin{equation}
    \pi(s) = \argmax_{a} [{r(s,a) + \gamma\sum_{s' \in S}p(s'|s,a)V(s')}]
\end{equation}

It is often helpful to know what the performance will be if we extract a greedy policy from an arbitrary value function. To see this, we introduce the notion of a Bellman residual.

Define the Bellman residual to be $(BV - V)$ and the Bellman error magnitude to be $||BV - V||$.

\begin{enumerate}
    \item[(d)] For what value function $V$ does the Bellman error magnitude $\|BV - V \|$ equal 0? Why? [2 pts]

    \begin{tcolorbox}[breakable]
    %%%%% Start of 3(d) %%%%%
    When the value function $V$ is a fixed point (i.e $V=V^{\pi}$ or $V=V^*$), we have \begin{align*}
        || BV - V || = || V - V || = 0
    \end{align*}
    %%%%% End of 3(d) %%%%%
    \end{tcolorbox}
    \newpage
    \item[(e)] Prove the following statements for an arbitrary value function $V$ and any policy $\pi$.  [5 pts]\\
    \textit{Hint:} Try leveraging the triangle inequality by inserting a zero term.
    \begin{equation}
        ||V - V^\pi|| \leq \frac{||V - B^\pi V||}{1-\gamma}
    \end{equation}
    \begin{equation}
        ||V - V^*|| \leq \frac{||V - BV||}{1-\gamma}
    \end{equation}

    \begin{tcolorbox}[breakable]
    %%%%% Start of 3(e) %%%%%
    Proof 1:
    \begin{align*}
        || V - V^{\pi} || &= \max_s | V(s) - V^{\pi} (s) - B^{\pi} V(s) + B^{\pi} V(s) |  \\
        &= \max_s | V(s) - B^{\pi} V(s) + B^{\pi} V(s) - V^{\pi} (s) | \\
        &\leq \max_s | V(s) - B^{\pi} V(s) | + \max_s |B^{\pi} V(s) - V^{\pi} (s) | \quad \text{triangle inequality} \\
        &= ||V-B^{\pi} V|| + ||B^{\pi} V - V^{\pi}|| \\
        &= ||V - B^{\pi} V|| + ||B^{\pi} V - B^{\pi} V^{\pi}|| \quad V^\pi \text{ is fixed point}\\
        &\leq || V - B^{\pi} V || + \gamma ||V - V^{\pi}|| \quad \text{part a} \\
        (1-\gamma) ||V- V^{\pi}|| &\leq ||V- B^{\pi} V|| \\
        ||V - V^\pi|| &\leq \frac{|| V - B^{\pi} V ||}{1-\gamma}
    \end{align*}

    In a similar vein, proof 2:
    \begin{align*}
        || V - V^* || &= \max_s | V(s) - V^* (s) - B V(s) + B V(s) |  \\
        &= \max_s | V(s) - B V(s) + B V(s) - V^* (s) | \\
        &\leq \max_s | V(s) - B V(s) | + \max_s |B V(s) - V^* (s) | \\
        &= ||V-B V|| + ||B V - V^*|| \\
        &= ||V - B V|| + ||B V - B V^*|| \\
        &\leq || V - B V || + \gamma ||V - V^*|| \\
        (1-\gamma) ||V- V^*|| &\leq ||V- B V|| \\
        ||V - V^*|| &\leq \frac{|| V - B V ||}{1-\gamma}
    \end{align*}
    %%%%% End of 3(e) %%%%%
    \end{tcolorbox}
    \newpage
\end{enumerate}

\noindent The result you proved in part (e) will be useful in proving a bound on the policy performance in the next few parts. Given the Bellman residual, we will now try to derive a bound on the policy performance, $V^\pi$.

\begin{enumerate}
    \item[(f)] Let $V$ be an arbitrary value function and $\pi$ be the greedy policy extracted from $V$. Let $\epsilon = ||BV-V||$ be the Bellman error magnitude for $V$. Prove the following for any state $s$. [5 pts]\\
    \textit{Hint:} Try to use the results from part (e).
    \begin{equation}
        V^\pi(s) \geq V^*(s) - \frac{2\epsilon}{1-\gamma}
    \end{equation}

    \begin{tcolorbox}[breakable]
    %%%%% Start of 3(f) %%%%%
    For this proof, I will equivalently prove that $V^* (s) - V^{\pi} (s) - \frac{2\epsilon}{1-\gamma} \leq 0$.
    \begin{align*}
        V^* (s) - V^{\pi} (s) - \frac{2\epsilon}{1-\gamma} &\leq \max_s |V^{\pi} (s) - V^* (s) | - \frac{2\epsilon}{1-\gamma} \\ 
        &= ||V^{\pi} - V + V - V^* || - \frac{2\epsilon}{1-\gamma}\\
        &\leq ||V^\pi - V|| + ||V-V^*|| - \frac{2\epsilon}{1-\gamma}  \quad \text{triangle inequality} \\
        &\leq \frac{||V-B^\pi V||}{1 - \gamma} + \frac{||V - BV||}{1-\gamma} - \frac{2 ||BV - V|| }{1-\gamma} \quad \text{part e}\\
        &= \frac{2 ||V-BV||}{1-\gamma} - \frac{2||BV-V||}{1-\gamma} \quad \text{as $B^\pi V = BV$ since $\pi$ is the greedy policy wrt $V$}\\
        &= 0
    \end{align*}
    %%%%% End of 3(f) %%%%%
    \end{tcolorbox}
    \newpage
    \item[(g)] Give an example real-world application or domain where having a lower bound on $V^\pi(s)$ would be useful. [2 pt]
    
    \begin{tcolorbox}[breakable]
    %%%%% Start of 3(g) %%%%%
    Lower bounds on $V^\pi (s)$ are important for applications where we definitely don't want low values, including high-risk applications such as autonomous driving where safety is extremely important.
    %%%%% End of 3(g) %%%%%
    \end{tcolorbox}\newpage

    \item[(h)] Suppose we have another value function $V'$ and extract its greedy policy $\pi'$.  $\|B V' - V' \| = \epsilon = \|B V - V\|$. Does the above lower bound imply that $V^\pi(s) = V^{\pi'}(s)$ at any $s$? [2 pts]
    
    \begin{tcolorbox}[breakable]
    %%%%% Start of 3(h) %%%%%
    No, the lower bound does not imply that $V^\pi(s) = V^{\pi'}(s)$ at any $s$. It only demonstrates that their lower bounds are the same, but not necessarily their exact values.
    %%%%% End of 3(h) %%%%%
    \end{tcolorbox}

\end{enumerate}


\newpage
\noindent {A little bit more notation:} define $V \leq V'$ if $\forall s$, $V(s) \leq V'(s)$. 
\\

\noindent What if our algorithm returns a $V$ that satisfies $V^* \leq V$? I.e., it returns a value function that is better than the optimal value function of the MDP. Once again, remember that $V$ can be any vector, not necessarily achievable in the MDP but we would still like to bound the performance of $V^\pi$ where $\pi$ is extracted from said $V$. We will show that if this condition is met, then we can achieve an even tighter bound on policy performance.



\begin{enumerate}
    \item[(i)] Using the same notation and setup as part (e), if $V^* \leq V$, show the following holds for any state $s$. [5 pts]\\
    \textit{Hint:} Recall that $\forall \pi$, $V^\pi \leq V^*$. (why?)
    \begin{equation}
        V^\pi(s) \geq V^*(s) - \frac{\epsilon}{1-\gamma}
    \end{equation}
     where $\epsilon = \|B V - V\|$  (as above) and the policy $\pi$ is the greedy policy induced by $V$. 
    
    \begin{tcolorbox}[breakable]
    %%%%% Start of 3(i) %%%%%
    I will equivalently prove $V^* (s) - V^{\pi} (s) - \frac{\epsilon}{1-\gamma} \leq 0$.
    \begin{align*}
        V^* (s) - V^{\pi} (s) - \frac{\epsilon}{1-\gamma} &\leq V(s) - V^{\pi}(s) - \frac{\epsilon}{1-\gamma} \quad \text{since $V^* \leq V$}\\
        &\leq ||V-V^{\pi}|| - \frac{\epsilon}{1-\gamma} \quad \text{definition of norm}\\
        &\leq \frac{||V-B^{\pi} V||}{1-\gamma} - \frac{||BV - V||}{1-\gamma}  \quad \text{part e}\\
        &= \frac{||V-B V||}{1-\gamma} - \frac{||BV - V||}{1-\gamma} \quad \text{as $B^\pi V = BV$, $\pi$ is the greedy policy wrt $V$} \\
        &= 0
    \end{align*}
    %%%%% End of 3(i) %%%%%
    \end{tcolorbox}
\end{enumerate}

\noindent \textbf{Intuition:} A useful way to interpret the results from parts (h) (and (i)) is based on the observation that a constant immediate reward of $r$ at every time-step leads to an overall discounted reward of $r + \gamma r + \gamma^2 r + \ldots = \frac{r}{1-\gamma}$. Thus, the above results say that a state value function $V$ with Bellman error magnitude $\epsilon$ yields a greedy policy whose reward per step (on average), differs from optimal by at most $2\epsilon$. So, if we develop an algorithm that reduces the Bellman residual, we're also able to bound the performance of the policy extracted from the value function outputted by that algorithm, which is very useful!
\\
\newpage
\noindent \textbf{Challenges:} Try to prove the following if you're interested. \textbf{These parts will not be graded.}

\begin{enumerate}
    \item[(j)] It's not easy to show that the condition $V^* \leq V$ holds because we often don't know $V^*$ of the MDP. Show that if $BV \leq V$ then $V^* \leq V$. Note that this sufficient condition is much easier to check and does not require knowledge of $V^*$. \\
    \textit{Hint}: Try to apply induction. What is $\lim\limits_{n \rightarrow \infty} B^n V$? 
    
    \begin{tcolorbox}[breakable]
    %%%%% Start of 3(j) %%%%%

    %%%%% End of 3(j) %%%%%
    \end{tcolorbox}
\newpage
\item[(k)] It is possible to make the bounds from parts (i) and (j) tighter. 
Let $V$ be an arbitrary value function and $\pi$ be the greedy policy extracted from $V$. Let $\epsilon = ||BV-V||$ be the Bellman error magnitude for $V$. Prove the following for any state $s$:
\begin{equation}
        V^\pi(s) \geq V^*(s) - \frac{2\gamma\epsilon}{1-\gamma}
\end{equation}
Further, if $V^* \leq V$, prove for any state $s$
\begin{equation}
        V^\pi(s) \geq V^*(s) - \frac{\gamma\epsilon}{1-\gamma}
\end{equation} 
    
    \begin{tcolorbox}[breakable]
    %%%%% Start of 3(k) %%%%%

    %%%%% End of 3(k) %%%%%
    \end{tcolorbox}
\end{enumerate}
\newpage

\noindent 


\newpage

\section{RiverSwim MDP [25 pts]}
Now you will implement value iteration and policy iteration for the RiverSwim environment (see picture below\footnote{Figure copied from \href{https://proceedings.neurips.cc/paper/2013/hash/6a5889bb0190d0211a991f47bb19a777-Abstract.html}{(Osband \& Van Roy, 2013)}.}) of \href{https://www.sciencedirect.com/science/article/pii/S0022000008000767}{(Strehl \& Littman, 2008)}.
\begin{figure}[h]
    \centering
    \includegraphics[width=\linewidth]{RiverSwim.png}
    \caption{The RiverSwim MDP where dashed and solid arrows represent transitions for the \textsc{LEFT} and \textsc{RIGHT} actions, respectively. The assignment uses a modified, customizable version of what is shown above where there are three different strengths (\textsc{WEAK}, \textsc{MEDIUM}, or \textsc{STRONG}) of the current (transition probabilities for being pushed back or successfully swimming \textsc{RIGHT}).}
    \label{fig:riverswim}
\end{figure}

\noindent \textbf{Setup:} This assignment needs to be completed with Python3 and \texttt{numpy}. 
\\

\noindent \textbf{Submission:} There is a \texttt{Makefile} provided that will help you submit the assignment. Please run \texttt{make clean} followed by \texttt{make submit} in the terminal and submit the resulting zip file on Gradescope.

\begin{enumerate}[label=(\alph*)]
\item \textbf{(coding)} Read through \texttt{vi\_and\_pi.py} and implement \texttt{bellman\_backup}. Return the value associated with a single Bellman backup performed for an input state-action pair. [4 pts]\\


\item \textbf{(coding)} Implement \texttt{policy\_evaluation}, \texttt{policy\_improvement} and \texttt{policy\_iteration} in  \texttt{vi\_and\_pi.py}. Return the optimal value function and the optimal policy. [8pts]\\


\item \textbf{(coding)} Implement \texttt{value\_iteration} in \texttt{vi\_and\_pi.py}. Return the optimal value function and the optimal policy. [8 pts]\\


\item \textbf{(written)} Run both methods on RiverSwim with a \textsc{weak} current strength and find the largest discount factor (\textbf{only} up to two decimal places) such that an optimal agent starting in the initial far-left state (state $s_1$ in Figure \ref{fig:riverswim}) \textbf{does not} swim up the river (that is, does not go \textsc{RIGHT}). Using the value you find, interpret why this behavior makes sense. Now repeat this for RiverSwim with \textsc{medium} and \textsc{strong} currents, respectively. Describe and explain the changes in optimal values and discount factors you obtain both quantitatively and qualitatively. [5 pts]\\ \\
\textit{Sanity Check:} For RiverSwim with a discount factor $\gamma = 0.99$ and a \textsc{weak} current, the values for the left-most and right-most states ($s_1$ and $s_6$ in Figure \ref{fig:riverswim} above) are \texttt{30.328} and \texttt{36.859} when computed with a tolerance of $0.001$. The value functions from VI and PI should be within error tolerance $0.001$ of these values. You can use this to verify your implementation. For grading purposes, we shall test your implementation against other hidden test cases as well.
 
\begin{tcolorbox}[breakable]
%%%%% Start of 4(d) %%%%%
WEAK: $\gamma=0.67$, optimal values = [0.015 0.033 0.092 0.257 0.717 1.993]. A $\gamma=0.67$ signifies that we care about the future a significant amount, but not a lot. This is the largest value such that we move to the left at state $s_1$, which makes sense because any larger value means we care considerably more about the future, which would incentivize the agent to move to the right.
    
MEDIUM: $\gamma=0.77$, optimal values = [0.022 0.035 0.095 0.275 0.798 2.314].
    
STRONG: $\gamma=0.93$, optimal values = [0.068 0.078 0.146 0.377 1.079 3.169].

As the current increases in strength, there are two main changes:
\begin{enumerate}
    \item A higher value of $\gamma$ is required for the agent to turn left at state $s_1$
    \item Values of states increase, especially later states (i.e. state $s_6$)
\end{enumerate}
This makes sense because with stronger currents, we have a higher probability of being pushed back to the left. Therefore, we need a higher value for $\gamma$ (how much we care about the future) to justify moving to the right from state $s_1$. The optimal values increase because of the increasing minimum $\gamma$ required.
%%%%% End of 4(d) %%%%%
\end{tcolorbox}

\end{enumerate}

\end{document}
